% --------------------------------------------------------------------------- %
% --------------------------------------------------------------------------- %
\chapter{Introduction}
\label{ch:introduction}

This dissertation documents a search for new physics in proton-proton collisions at the Harge Hadron Collider, wth data collected by the Compact Muon Solenoid at a center-of-mass energy of 13 TeV. The state of particle physics and motivations for such a search are summarized in Chapter \ref{ch:intro}, including the current understanding of the Standard Model of particle physics and theorized extensions of the Standard Model which motivate these physics searches.

Chapter \ref{ch:detector} describes the experimental apparatus used to collect the data analyzed in this analysis. Section \ref{sec:lhc} describes the physics of the Large Hadron Collider which delivers protons to the collision point, and section \ref{sec:cms} details the different subsystems of the Compact Muon Solenoid detector which collect particle data from collisions. The data are reconstructed into abstract physics objects as described in section \ref{sec:physicsobjects}, which are subsequently analyzed in this search for new physics.

Chapters \ref{ch:analysis}, \ref{ch:bkgs}, and \ref{ch:results} describe the analysis design and execution. In chapter \ref{ch:analysis}, the overall analysis strategy and signal definitions are discussed, and chapter \ref{ch:bkgs} describes the background predictions in detail. Chapter \ref{ch:results} presents the results of the search, including interpretations constraining parameters of new physics models.

Finally, chapter \ref{ch:soft} presents an extension of the all-hadronic search that targets specific signatures motivated by new physics models and limitations of the typical all-hadronic analysis. A pilot analysis of this type was presented in \cite{CMS-PAS-SUS-16-011}.

The all-hadronic analyses detailed in this dissertation were published in \cite{mt2_2016} and \cite{mt2_2017}.

% --------------------------------------------------------------------------- %
% --------------------------------------------------------------------------- %
