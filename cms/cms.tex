% --------------------------------------------------------------------------- %
% --------------------------------------------------------------------------- %
\chapter{The LHC Accelerator and CMS Experiment}
\label {ch:cms}

\section{The Large Hadron Collider}
The Large Hadron Collider (LHC) is the largest particle accelerator in the world.
It is designed to collide protons against other protons with a center of mass energy up to 14 TeV.
The first collisions used for physics happened in March 2010, where the LHC was running at a center of mass energy of 7 TeV.
The LHC was restarted the following year and operated at a higher energy of 8 TeV
eventually delivering about four times the amount of data that was taken at 7 TeV.
After the successful first run ended, the machine was shut down in order to upgrade many components.
The next time the LHC was run was in May 2015, where protons were successfully collided at a center of mass energy of 13 TeV.
The analysis shown in this thesis is based on the data taken during the 2015 running period.

In order to achieve such large center of mass energies,
the protons are accelerated in a circular beam pipe which is 27 km long and and 100 m underground.
Since the pipe is circular,
it's possible to accelerate two beams of protons in opposite directions and repeatedly collide them at various points around the ring.
There are 4 points along the ring where protons are made to collide and the resulting collisions are recorded by various experiments as seen in figure~\ref{fig:lhcunderground}.
The Compact Muon Solenoid (CMS) experiment is where the data was taken used for the results shown in this thesis.

\begin{figure}[!ht]
\begin{center}
\includegraphics[width=0.8\textwidth]{cms/figs/lhc-underground.jpeg}
\caption{ The LHC can be seen along with the 4 main experiments at different points along the beamline.
\label{fig:lhcunderground}
}
\end{center}
\end{figure}

\section{The Compact Muon Solenoid}
The CMS detector consists of a triggering system and many sub detectors,
where measurements taken using these sub detectors are used as input to the triggering system.
When protons are collided at the interaction point within CMS, approximately N collisions happen per second.
The volume of collisions is so high that it is impossible to record all of the data all of the time.
The way this is controlled is by the use of triggers, the level 1 (L1) trigger and the high level triggers (HLT).
The triggers are designed such that the full event information is stored when all the requirements of a trigger are met.
For some physics processes, trigger rates are much higher than can be afforded by the allowed budget of the CMS experiment.
These processes can still be studied by prescaling the triggers meaning the full information is saved once for every set number of events.
As the prescale is increased, the rate of the trigger is reduced.

The CMS sub detector components are layered in such a way that one can take advantage of different particle interactions with different materials.
The innermost layer is the tracker (decribed in section~\ref{subs:tracker}) which is used to measure the charge and and momentum of charged particles.
When a charged particle passes through the silicon layers in the pixel detector, a small amount of energy is deposited.
The deposit is lager enough that a hit will be registered by the electronics, but small enough that the trajectory of the particle will be minimally affected.
Neutral particles do not interact with the tracker.

The next layer is the Electromagnetic Calorimeter (decribed in section~\ref{subs:ECAL}) which is used to contain and measure the energy of electrons and photons.
The Electromagnetic Calorimeter (ECAL) is made of lead tungstate ($\mathrm{PbWO_{4}}$) scintilating crystals.
The material has a radiation length of 0.89 cm and Moli\`ere radius ($\mathrm{R_{M} \simeq 2.19~cm}$).
Electrons and photons that enter the ECAL lose energy due to showering, and the depth of the ECAL is sufficient enought to fully contain these particles.

Outside of the ECAL is the Hadronic Calorimeter (decribed in section~\ref{subs:HCAL}) which is used to contain and measure the energy of hadronic particles.
The Hadronic Calorimeter (HCAL) is designed in such a way that hadronic particles incident on the HCAL lose energy to showering.
The goal is to fully contain these particles, although sometimes particles have sufficient energy that they are able to escape before losing sufficient energy to showering.
This phenomenon is known as punch-through.

Beyond the HCAL is the solenoid which produces a magnetic field of 3.8 T at the center of the detector.
This magnetic field is used to bend the tracks of charged particles in order to measure the momentum following the equation~\ref{eqn:bfieldmomentum},
where q is the particle charge, r is the radius of curvature, and B is the magnetic field magnitude.

\begin{equation}
  \label{eqn:bfieldmomentum}
p = qrB
\end{equation}

The only particles that should pass beyond the HCAL are muons and neutrinos.
Neutrinos are measured indirectly by assuming that the momentum in the direction transverse to the beam is zero at the beginning of each collision,
and then calculating the sum of the total transverse momentum of all particles directly measuered in the detector.
Since Neutrinos pass through the detector with no interactions, they contribute to missing transverse momentum, which will be referred to as (\MET) or MET.

Lastly, the muon system (decribed in section~\ref{subs:MUON}) is used to measure the charge and and momentum of muons.
At the energies that muons are most commonly produced in collisions at the LHC, muons are minimum ionizing particles (MIPs)
meaning they do not lose very much energy as they pass through the detector components.

\subsection{Silicon Tracker}
\label {subs:tracker}
figure~\ref{fig:tracker}

(NEED TO ADD SECTION ON TRACKER)

\begin{figure}[!htb]
\begin{center}
\includegraphics[width=0.8\textwidth]{cms/figs/Barrel.pdf}
\caption{
  A cross-sectional in the x-y plane of the CMS tracking system is shown here.
  The tracker provides excellent coverage for reconstruction of tracks from charged particles for $|\eta| < $2.5.
\label{fig:tracker}
}
\end{center}
\end{figure}

\subsection{Electromagnetic Calorimeter}
\label {subs:ECAL}
The ECAL is split into two regions, the ECAL barrel (EB), and ECAL endcap (EE).
The EB occupies a region in the detector within $|\eta| < 1.479$, and the EE occupies the region with $1.479 < |\eta| < 3.0$.
The geometry can be seen in figure~\ref{fig:ecal_eta}.

\begin{figure}[!htb]
\begin{center}
\includegraphics[width=0.8\textwidth]{cms/figs/Figures_Experimental_Apparatus_ECALRapidity.pdf}
\caption{ A cross-sectional view of the CMS ECAL is shown in this figure, with values of $\eta$ shown that determine the coverage of each subdetector.
\label{fig:ecal_eta}
}
\end{center}
\end{figure}

\clearpage

As written previously, the ECAL is made of lead tungstate crystals.
These crystals act as absorbers as well as scintilators.
What this means is that the when electrons or photons are incident on the crystals, they lose energy due to showering.
The showering process can be simply described as a combination of electromagnetic scattering, and electron-positron pair production.
Leading order diagrams of these processes are shown in figures~\ref{fig:pair_production}~and~\ref{fig:photon_brem}.

\begin{figure}[!ht]
\begin{center}
\includegraphics[width=0.8\textwidth]{cms/figs/Pair_Production.png}
\caption{ A feynman diagram is shown depicting electron-positron pair production from a photon initial state.
\label{fig:pair_production}
}
\end{center}
\end{figure}

\begin{figure}[!ht]
\begin{center}
\includegraphics[width=0.8\textwidth]{cms/figs/photon_brem.png}
\caption{
  A feynman diagram is shown of a process where an electron scatters off of an incoming photon,
  and then radiates a photon through bremsstrahlung.
\label{fig:photon_brem}
}
\end{center}
\end{figure}

As the initial particle passes through the crystal, it loses energy due to these processes, and this energy is converted to light by the scintilating properties of the crystal.
This light is read out using a avalanche photo-diode (APDs) for each crystal in the EB, and a vacuum phototriode for each crystal in the EE.
The amount of energy absorbed by the crystals can be parameterized by a quantity known as the radiation length ($X_{0}$).
The expected amount of energy left (E(x)) after a particle with initial energy ($E_{0}$)
travels a distance x through an absorbing material is shown in equation~\ref{eqn:radiation_length}.
The CMS ECAL has a depth of \~25$X_{0}$ at small (O(0)) and large (1.6-3.0) values of $\eta$.
In between these regions, the depth is larger due to the barrel-like geometry of the ECAL.
Because of this depth, electron and photon energies are measured very well in the ECAL.
There is one exception where the depth of the ECAL is small, and this is where the barrel and endcap meet.
This leads to poor reconstruction of electrons and photons in this region.

\begin{equation}
  \label{eqn:radiation_length}
  E(x) = E_{0}e^{\frac{-x}{X_{0}}}  
\end{equation}

\subsection{Hadronic Calorimeter}
\label {subs:HCAL}
The Hadronic Calorimeter (HCAL) is used to measure energies of hadronic particles, such as pions, gluons, protons, kaons and neutrons.
It is a sampling calorimeter meaning it has many layers of absorber material (brass) and scintilator material (plastic).
As hadronic particles pass through the HCAL, they interacts with the absorber layers by showering, meaning they decay to multiple lower energy particles.
As these particles shower and the constituents pass through the scintilator, photons are emitted by the scintilating material and then measured by Hybrid-Photo Diodes (HPDs).

Approximately a third of the particles produced in a hadronic shower are $\pi_{0}$s, which immediately decay to two photons most of the time.
These particles are mostly stopped in the absorbing layer making it hard to accurately determine the energy lost due to this effect.
Some detectors however are built to be able to accurately measure the energy lost to electromagnetic effects in the shower, and these are known as ``compensating'' calorimeters.
When a particle is stopped by the HCAL, the energy deposited in all the layers is integrated over a full segment of the HCAL, and used to determine an incident particles energy.
The HCAL at CMS is not able to compensate for these effects, so it must be calibrated offline using particles with a well-known initial energy.

The HCAL geometry is such that it is fully contained within the solenoid.
It is divided into towers, each of which lies behind an integer number of ECAL crystals.
This helps in calibrating the energies of incident particles during reconstruction.
The number of interaction lengths ($\lambda_{int}$) in the HCAL starts at about $\lambda_{int} = 7$ at $\eta = 0$, and increases with eta to the end of the barrel up to $\lambda_{int} = 11$.
There is then a slight decrease in the endcap of $\lambda_{int} = 10$.
The interaction length is used to characterize the longitudinal and transverse profile of hadronic showers.
The expected number of secondary particles produced ($N(x)$)
by a number of particles traveling through the HCAL ($N_{0}$)
is written in equation~\ref{eqn:interaction_length} as a function of $\lambda_{int}$ and the distance traveled (x).

\begin{equation}
  \label{eqn:interaction_length}
  N(x) = N_{0}e^{\frac{-x}{\lambda_{int}}}  
\end{equation}

\subsection{Muon System}
\label {subs:MUON}
The muon system is made up of 1400 chambers divided into 3 categories: 610 resistive plate chambers (RPCs), 250 drift tubes (DTs), and 540 cathode strip chambers (CSCs).
It is the only detector subsystem which lies completely outside of the solenoid.
The information from these systems is used sometimes for triggering on events as well as help to identify muons and measure the muon properties.
Each system is located in different regions in the detector, with some systems overlapping others.
The DTs cover a region of $|\eta| < 1.2$, the CSCs cover a region of $0.9 < |\eta| < 2.4$, and the RPCs cover a region of $|\eta| < 2.1$.

Each DT is 4 cm wide with a length of wire running all the way down the middle.
A diagram of a DT is shown in figure~\ref{fig:DT}.
The DTs are individually filled with a mixture of Ar and $\mathrm{CO_{2}}$ gas, and when muons pass through the DTs, this gas is ionized.
A voltage difference is maintained between the wire and the outside of the DT such that when the gas is ionized,
the ionized particles drift towards the charged components causing a voltage spike.
The DTs are then arranged in such a way that a muon passing through the region where these are located will leave a hit in multiple DTs.
The muon's trajectory can then be reconstructed using this information.

\begin{figure}[!htb]
  \begin{center}
    \includegraphics[width=0.8\textwidth]{cms/figs/DT.pdf}
    \caption{
      \label{fig:DT}
      A cartoon depiction of a drift tube is shown in this figure. 
    }
  \end{center}
\end{figure}

There are 540 CSCs shown in figure~\ref{fig:CSC}.

\begin{figure}[!htb]
\begin{center}
\includegraphics[width=0.8\textwidth]{cms/figs/CSC.pdf}
\caption{ A cartoon depiction of a cathode strip chamber is shown in this figure. 
\label{fig:CSC}
}
\end{center}
\end{figure}

There are 610 resistive plate chambers (RPCs), an example of which is shown in figure~\ref{fig:RPC}.

\begin{figure}[!htb]
  \begin{center}
    \includegraphics[width=0.8\textwidth]{cms/figs/RPClayers.pdf}
    \caption{
      \label{fig:RPC}
      A cartoon depiction of a resistive plate chamber is shown in this figure. 
    }
  \end{center}
\end{figure}


\section{Physics objects}
The data taken by the CMS detector has to be processed in such a way that all the physics processes that happen in a single collision can be reconstructed.
Each reconstructed collision is called an event.
In each event, there are multiple physics objects that are reconstructed.
The main objects that are pertinent to this analysis are electrons, muons, jets, and photons.
Each of these objects has a unique signature in the detector, but it is still possible for this signature to be faked.
The reconstruction of these objects is discussed in detail in the following sections.

\subsection{Particle Flow}
\label{subs:particleflow}
In order to classify objects, an algorithm named particle flow (PF) is used~\cite{pfReco}. 
The main goal when using the pf algorithm is to account for all energies measured in the detector.
The way this is done is by clustering energies measured across all subdetectors into separate PF candidates, and classifying these cadidates.
Once an object is classified, the clustering is done again with all the energy associated with the classified object removed.
This process is repeated until all measured energies are accounted for.
The objects that are identified by this algorithm are listed below.
In order to further reduce fakes from contaminating our signal region, we make additional cuts to help classify objects.
These cuts are described in detail in the following section for each object we are interested in.

\begin{itemize}
\item muon          
\item photon        
\item charged hadron
\item neutral hadron
\item electron      
\end{itemize}

\subsection{Vertex Determination and Pileup}
\label{ssec:vtxandpileup}
The LHC collides protons in large bunches to maximize the probability of a hard collision.
Multiple pairs of protons can interact at each bunch crossing, and the number of interactions depends on the beam density.
It is important to identify which proton-pair interaction is responsible triggering the event being studied.
In this analysis, the primary vertex is chosen using track information.
Every track is associated with a unique vertex,
and the primary vertex is chosen by finding the vertex with the largest sum of $\mathrm{(p_{T})^{2}}$ of tracks associated with that vertex.
Neutral candidates do not have associated tracks, so they are redefined to always be from the primary vertex.
The energy of objects affected by this reassociation is recalibrated, and this calibration is described in section~\ref{ssec:jets}.

Pileup is defined as energy in the detector which does not come from the interaction being studied.
The majority of energy from pileup comes from proton-proton interactions other than primary interaction.
These soft interactions produce predominantly hadronic final states, for example two jets from  gluon splitting.
Due to these final states being hadronic, the overall energy deposited in the detector is balanced.
However, there can still be small contributions to \MET\ from to pileup due to the poor resolution of jets in CMS.
A calibration is applied to account for pileup according to the method described in section~\ref{sec:t1met}.

\subsection{Isolation}
\label{ssec:isolation}
Isolation is a concept that is very important when identifying leptons, and photons.
It can be simply described as the total energy in the detector near an object of interest.
The distance between two objects ($\Delta$R) is defined by equation~\ref{eqn:DR},
where $\phi$ is the angle measured in plane transverse to the beam direction,
and $\eta$ is the psuedorapidity defined by equation~\ref{eqn:psuedorapidity}.
In the equation for $\eta$, the variable $\theta$ is defined as the angle measured from the beam axis to the axis of the transverse plane to the beam axis.
In particle physics, $\eta$ is preferred over $\theta$ when describing a particle's momentum along the beam axis because changes in $\eta$ are lorentz invariant,
whereas the same is not true for changes in $\theta$.

\begin{equation}
  \label{eqn:DR}
  \Delta R = \sqrt{\Delta\eta^{2}+\Delta\phi^{2}}
\end{equation}

\begin{equation}
  \label{eqn:psuedorapidity}
  \eta = -\ln(\mathrm{tan}(\frac{\theta}{2}))
\end{equation}

The choice if what to include when calculating isolation is defined differently for each object and will be described in the following sections.

\subsection{leptons and photons}
\label{ssec:lepsandphots}
Z bosons can decay leptonically to $e\bar{e}$, $\mu\bar{\mu}$, and $\tau\bar{\tau}$~pairs,
but identifying $\tau$s is very complicated in CMS, and their resolution is not as good as electrons and muons.
In order to keep the analysis simple, this search is done in final states with $Z\rightarrow ee$ and $Z\rightarrow\mu\bar{\mu}$ only,
and when referencing leptons, only electrons and muons are considered.
It is also important to be able to identify photons in order to measure one of the main backgrounds.

The depth of the ECAL is \~25 radiation lengths which means electrons and photons lose almost all of their energy in the ECAL.
Another thing that helps to distinguish electrons from photons is that electrons will deposit energy in the silicon tracker whereas photons will not.
Therefore, a track can be matched to energy deposits in the ECAL to help identify electrons.
Similarly, a track found to correspond to a hit in the ECAL can be used to help determined that the energy deposited in the ECAL was not from a photon.

Muons have a low interaction cross section with the materials used to build CMS in the energy range they are most produced in collisions at the LHC.
They are referred to as minimum ionizing particles, and the stopping power of copper on muons can be seen in figure~\ref{fig:muonenergyloss}.
Therefore, muons do not deposit very much energy in any of the calorimeters.
A different approach is taken to identify and measure muons in the CMS detector.
Since muons are charged, they leave hits in the silicon detector which can be reconstructed as a track.
The ECAL and HCAL are designed to contain all measureable energy,
but due to the minimum ionizing nature of muons, muons are able to pass through these subdetectors without depositing much energy.
The muon subdetector is the outermost layer of the CMS detector, and it is designed to identify muons.
Muon tracks can be measured independently in the muon subdetector,
and the measurement is matched to a set of hits in the tracker to determine the final muon energy.

The main thing all of these objects have in common is they are produced in interactions that do not have a hadronic component.
Quantifying this energy is done by calculation the particles isolation, as described in section~\ref{ssec:isolation},
and is used to discriminate against unwanted physics processes.

\begin{figure}[!htb]
  \begin{center}
    \includegraphics[width=0.8\textwidth]{cms/figs/muon_energy_loss.pdf}
    \caption{
      \label{fig:muonenergyloss}
      Stopping power of copper as a function of muon momentum. Typical momenta of muons in this analysis are between 20-200 GeV. 
    }
  \end{center}
\end{figure}


\subsection{jets}
\label{ssec:jets}
A jet is an object used to represent a single parton, and it can be simply described as a substantial amount of hadronic activity concentrated in a single region of the detector.
The region is defined using the ``anti-$\mathrm{k_{T}}$'' (ak) algorithm~\cite{antikt} with a jet radius of 0.4.
For jets in this analysis, the charged candidates not associated with the primary vertex are removed before clustering. 
This removes some unwanted energy from pileup, and the jets are then calibrated in order to more accurately reflect the energy of the particles they represent.

The jet energy corrections are done in multiple levels, labeled as L1L2L3.
A detailed procedure of how the corrections are applied can be found here~\cite{CMS-DP-2015-044}.
Jets are corrected such that the calibrated energy is as close to the parton that the jet came from as possible.
The corrections are derived using MC, and each level corrects for different effects in the following way:
L1 corrects for pileup,
L2 corrects for different responses in the detector as a function of $\eta$,
and L3 corrects the overall energy scale.
The L1 correction is done by assuming a flat overall energy density in the detector which is calculated per event,
and then subtracting this energy from the jet using the jet area to determine the overall energy contributed from pileup.
The L2 correction is done by selecting di-jet events where one of the jets is central and the second jet is non-central $\eta$.
Scale factors are then derived to correct the energy measurement for the non-central jet in separate regions of $\eta$.
The L3 correction is done by deriving scale factors as a function of jet \pt\ which correct the jet energy to match the true energy of the parton that the jet came from.

In addition to the L1L2L3 corrections, residual corrections are derived in data control regions to correct for differences in the detector response to jets between data and MC.
This is done in a region with at least one photon or a Z boson which decays to two leptons, where the boson recoils off of a jet.
The photon and leptons are measured with better resolution than the jets, so scale factors are derived as a function of \pt\ to correct the detector response of jets.
The full description of cuts used to define jets for the analysis is defined in sub-section~\ref{ssec:jetsel}.
