% No symbols, formulas, superscripts, or Greek letters are allowed
% in your title.
\title{A Search for New Physics Using the Stransverse Mass Variable in All-Hadronic Final States Produced in Proton-Proton Collisions With a Center of Mass Energy of 13 TeV}

\author{Mark Derdzinski}
\degreeyear{2018}

% Master's Degree theses will NOT be formatted properly with this file.
\degreetitle{Doctor of Philosophy} 

\field{Physics}
\chair{Professor Frank W\"urthwein}

%  The rest of the committee members  must be alphabetized by last name.
\othermembers{
Professor Avraham Yagil, Cochair\\ 
Professor Farhat Beg\\
Professor Ian Galton\\
Professor Benjamin Grinstein\\
}
\numberofmembers{5} % |chair| + |cochair| + |othermembers|


\begin{frontmatter}
\makefrontmatter

%% ----------------------------------------------------------------------- %%
%% DEDICATION
%% ----------------------------------------------------------------------- %%

\begin{dedication}                                                             
To my family, who has always supported me in all my pursuits.
\end{dedication}                                                               
\clearpage 

%% ----------------------------------------------------------------------- %%
%% EPIGRAPH
%% ----------------------------------------------------------------------- %%

%  The same choices that applied to the dedication apply here.
% \begin{epigraph} % The style file will position the text for you.              
%   \it{Mon seul d\'esir est de m'enrichir de nouvelles pens\'ees exaltantes.} \\
%   ---Ren\'e Magritte
% \end{epigraph}                                                                 
\begin{myepigraph} % You position the text yourself.                           
  \vfil                                                                        
  \vfil 
\begin{centering}
\noindent{\it For knowledge comes slowly, and when it comes, it is often at great personal expense.} 
\end{centering}
  \vfil 
  %% \noindent {\it translation} \hfill 
  \vfil 
  \hfill --- Paul Auster, {\it Ghosts}
  \vfil 
\end{myepigraph}                                                               

\tableofcontents
\listoffigures  % Uncomment if you have any Figures                            
\listoftables   % Uncomment if you have any Tables                             

\begin{acknowledgements}                                                       
First and foremost I want to acknowledge my entire family, namely my parents Anna Derdzinska and Kris Derdzinski, and my sisters, Andrea and Magdalena. It was at home I first discovered my love of science, and without their constant support and encouragement I would not be where I am today.

I would like to thank the entire Surf \& Turf group, without whom I could not have completed this research. The support network provided by my colleagues both professionally and personally not only made this work possible, but made my life in particle physics enjoyable.

Both of my graduate advisors, Frank W\"urthwein and Avi Yagil, were incredibly supportive faculty mentors. Their guidance helped shape me as a physicist, and their patience granted me the opportunity to explore my other passions at the same time.

I am deeply indebted to my post-doc mentor, Giovanni Zevi Della Porta. My daily interactions with Gio were not just integral integral to this work, but personally rewarding. I have never met a more collegial, supportive, and patient physicist, and could not have wished for a better mentor as a graduate student.

I am very grateful to my undergraduate advisor, Yury Kolomensky, for first indulging my curiosity in particle physics. Yury gave me the opportunity to explore my interests in the field, and his thoughtful mentorship set me on the path to where I am today.

Finally, I would like to acknowledge the thousands of faculty, students, and staff that make experiments like the LHC and CMS possible. This work could not have happened without all the unnamed scientists that laid the foundation before me, and I am eternally grateful for being given the opportunity to contribute to the legacy of our field. 

In particular, I would like to congratulate everyone in the CERN accelerator departments for the excellent performance of the LHC and thank the technical and administrative staffs at CERN and at other CMS institutes for their contributions to the success of the CMS effort. In addition, I gratefully acknowledge the computing centers and personnel of the Worldwide LHC Computing Grid for delivering so effectively the computing infrastructure essential to our analyses. Lastly, I acknowledge the enduring support for the construction and operation of the LHC and the CMS detector provided by the following funding agencies: BMWFW and FWF (Austria); FNRS and FWO (Belgium); CNPq, CAPES, FAPERJ, and FAPESP (Brazil); MES (Bulgaria); CERN; CAS, MoST, and NSFC (China); COLCIENCIAS (Colombia); MSES and CSF (Croatia); RPF (Cyprus); SENESCYT (Ecuador); MoER, ERC IUT, and ERDF (Estonia); Academy of Finland, MEC, and HIP (Finland); CEA and CNRS/IN2P3 (France); BMBF, DFG, and HGF (Germany); GSRT (Greece); OTKA and NIH (Hungary); DAE and DST (India); IPM (Iran); SFI (Ireland); INFN (Italy); MSIP and NRF (Republic of Korea); LAS (Lithuania); MOE and UM (Malaysia); BUAP, CINVESTAV, CONACYT, LNS, SEP, and UASLP-FAI (Mexico); MBIE (New Zealand); PAEC (Pakistan); MSHE and NSC (Poland); FCT (Portugal); JINR (Dubna); MON, RosAtom, RAS, RFBR and RAEP (Russia); MESTD (Serbia); SEIDI, CPAN, PCTI and FEDER (Spain); Swiss Funding Agencies (Switzerland); MST (Taipei); ThEPCenter, IPST, STAR, and NSTDA (Thailand); TUBITAK and TAEK (Turkey); NASU and SFFR (Ukraine); STFC (United Kingdom); DOE and NSF (USA).

\end{acknowledgements}                                                         

\begin{vitapage}                                                               
\begin{vita}                                                                   
  \item[2011] B.~A. in Physics and Mathematics, University of California, Berkeley                                                    
  \item[2015] M.~S. in Physics, University of California, San Diego                                                    
  \item[2018] Ph.~D. in Physics, University of California, San Diego    
\end{vita}                                                                     
\begin{publications}                                                           
\item \textbf{Search for new physics with the $M_{T2}$ variable in all-jets final states produced in pp collisions at $\sqrt{s}=13~\mathrm{TeV}$}, {\it CMS Collaboration},  \href{http://dx.doi.org/10.1007/JHEP10(2016)006}{\textcolor{blue}{J. High Energ. Phys. \textbf{10} (2016) 006}}, {\tt arXiv:\href{http://arxiv.org/abs/1603.04053}{\textcolor{black}{1603.04053 [hep-ex]}}}
\item \textbf{Search for new physics in the one soft lepton final state using 2015 data at $\sqrt{s}=13~\mathrm{TeV}$}, {\it CMS Collaboration}, Physics Analysis Summary (2016), CMS-PAS-SUS-16-011, \textcolor{black}{\href{https://cds.cern.ch/record/2161097}{cds.cern.ch/record/2161097}}
\item \textbf{Search for new phenomena with the $M_{T2}$ variable in the all-hadronic final state produced in proton-proton collisions at $\sqrt{s}=13~\mathrm{TeV}$}, {\it CMS Collaboration}, \href{http://dx.doi.org/10.1140/epjc/s10052-017-5267-x}{\textcolor{blue}{Eur. Phys. J. C {\bf77} (2017) no. 10, 710}}, {\tt arXiv:\href{http://arxiv.org/abs/1705.04650}{\textcolor{black}{1705.04650 [hep-ex]}}}

\end{publications}                                                             
\end{vitapage}                                                                 
                                                                               

%% ABSTRACT
%  Doctoral dissertation abstracts should not exceed 350 words. 
%   The abstract may continue to a second page if necessary.
\begin{abstract}
A search for physics beyond the Standard Model (SM) is performed in events with final states including hadronic activity, missing energy, and significant momentum imbalance as measured with the \mttwo variable. The results are based on data collected by the Compact Muon Solenoid detector at the Large Hadron Collider, and correspond to a total integrated luminosity of 35.9\fbinv of proton-proton collisions at a center-of-mass energy of 13\TeV. No significant excess above the predicted SM background is observed. The results are interpreted as 95\% confidence-level exclusion limits on the masses of hypothesized particles in a variety of simplified models of $R$-parity conserving supersymmetry (SUSY). Additional techniques for extending the search to target final states with low-momentum leptons is discussed, interpreted in in the context of SUSY models with ewkino decays, and compared to the exclusion strength of a typical all-hadronic targeting such models.
\end{abstract}


\end{frontmatter}
